\documentclass[10pt, xcolor=dvisnames]{beamer}

\usepackage{pgfpages}
\usepackage{tikz}
\usepackage{color}
\usepackage[dvisnames]{xcolor}
\usepackage{comment}
\usepackage{verbatim}
\usepackage{graphicx}

\usepackage[absolute,overlay]{textpos}

\setbeamersize{text margin left=40pt,text margin right=40pt} 

%\newcommand{\topline}{
  %\tikz[remember picture,overlay] {
    %\draw[black] ([yshift=-1.2cm,xshift=30]current page.north west)
    %\draw[black] ([yshift=-1.25cm,xshift=10]current page.north west)
             %-- ([yshift=-1.25cm,xshift=\paperwidth-10]current page.north west);}}
       %      -- ([yshift=-1.2cm,xshift=\paperwidth-30]current page.north west);
       %\node[black, align=center] at (\paperwidth/2-20,0) {some text};
      % }}
      
      
      % \node at (current page.north east) [left,yshift=-0.15\textheight] {\bf cat}; 
      
%\addtobeamertemplate{frametitle}{}{\tikz[overlay, remember picture]{\draw[black] ([yshift=-1.05cm,xshift=30]current page.north west)  -- ([yshift=-1.05cm,xshift=\paperwidth-30]current page.north west);}}
%\addtobeamertemplate{frametitle}{}{\tikz[overlay, remember picture]{\node ([yshift=-0.9cm,xshift=30]current page.north west) {\tiny Python Workshop};\draw[black] ([yshift=-0.9cm,xshift=30]current page.north west)  -- ([yshift=-0.9cm,xshift=\paperwidth-30]current page.north west);}}
\addtobeamertemplate{frametitle}{}{\tikz[overlay, remember picture]{\draw[black] ([yshift=-0.9cm,xshift=30]current page.north west)  -- ([yshift=-0.9cm,xshift=\paperwidth-30]current page.north west);}}
%\addtobeamertemplate{framtitle}{\vspace*{-20pt}}{\vspace*{-20pt}}



%\everymath{\color{red}}
\everydisplay{\color{red}}


% Give a slight yellow tint to the notes page
%\setbeamertemplate{note page}{\pagecolor{yellow!5}\insertnote}
%\usepackage{palatino}

\title{Python Workshop}
\author{Gabriel Gaudreault\\ Concordia University}
\date{\small \today}

\begin{document}
\maketitle




\begin{frame}{{\tiny \ \\\vspace{-13pt} \ \ \ \ \ \ \ \ \ \ \ Python Workshop}\\ \centerline{Today}}
\vspace*{-50pt}

Hello\\
\ \\

\begin{itemize}
\item Intro to Programming
\item Syntax of Python
\item Demos
\item Particularities of the Language
\item Extra Resources
\end{itemize}


\end{frame}







\begin{frame}{{\tiny \ \\\vspace{-13pt} \ \ \ \ \ \ \ \ \ \ \ Python Workshop}\\ \centerline{Why Python?}}
\vspace*{-50pt}

\begin{itemize}
\item Easy to Learn
\item More Natural Syntax: no ``;", fewer ``\{", ``\}"
\item No Pointers
\item Recent
\item Dynamically Typed
\item Automatic Memory Management
\item Lots of Libraries
\item Multi-Paradigm: Imperative, Object-Oriented, Functional, Scripting, Web
\item High-Level
\item Big Community
\end{itemize}


\end{frame}



\begin{frame}{{\tiny \ \\\vspace{-13pt} \ \ \ \ \ \ \ \ \ \ \ Python Workshop}\\ \centerline{Why Not Python?}}
\vspace*{-50pt}

See previous slide


\end{frame}



\begin{frame}{{\tiny \ \\\vspace{-13pt} \ \ \ \ \ \ \ \ \ \ \ Python Workshop}\\ \centerline{Python}}
\vspace*{-50pt}

{\small
\begin{tabular}{lll}
Interactive & Interpreted & Script\\
\hline
\texttt{Gabriel\$ python} & \texttt{Gabriel\$ vi code.py}& \texttt{Gabriel\$ vi code.py}\\
\texttt{>>> fav\_number = 3} & \texttt{Gabriel\$ python code.py} & \texttt{chmod +x code.py}\\
\texttt{>>> 2 + fav\_number} & & \texttt{Gabriel\$ ./code.py} \\
\texttt{5}\\
\texttt{>>>}\\
\end{tabular}
}

\ \\
For script mode, include \texttt{\#!/usr/bin/python} at top of the code

\ \\
``Main" function in Python : 

\texttt{
if \_\_name\_\_ == "\_\_main\_\_":\\
\ \ \ \ \ \ starting\_function()
}



%\includegraphics[width=3cm]{interactive}
%\includegraphics[width=6cm]{interpreted}
%\includegraphics[width=3cm, trim =100 50 50 50]{script1}
%\includegraphics[width=3cm]{script2}


\end{frame}



\begin{frame}{{\tiny \ \\\vspace{-13pt} \ \ \ \ \ \ \ \ \ \ \ Python Workshop}\\ \centerline{First Program}}
\vspace*{-50pt}

Printing salutations

\ \\
\ \\
\centerline{DEMO: HELLO + INTERACTIVE}

\end{frame}






\begin{frame}{{\tiny \ \\\vspace{-13pt} \ \ \ \ \ \ \ \ \ \ \ Python Workshop}\\ \centerline{Standard Data Types}}
\vspace*{-50pt}

\begin{itemize}
\item *Boolean
\item Numbers (integers, float, long, complex)
\item Strings
\item List
\item Tuple
\item Dictionary
\end{itemize}


\end{frame}



\begin{frame}{{\tiny \ \\\vspace{-13pt} \ \ \ \ \ \ \ \ \ \ \ Python Workshop}\\ \centerline{Variables}}
\vspace*{-50pt}

Unlike other languages, Python is dynamically typed, which means that in a lot of cases types do not have to be explicitly stated, for example in variable instantiation or when defining a function\\

\ \\
\ \\
\centerline{DEMO VARIABLES}


\end{frame}




\begin{frame}{{\tiny \ \\\vspace{-13pt} \ \ \ \ \ \ \ \ \ \ \ Python Workshop}\\ \centerline{Numbers}}
\vspace*{-50pt}

Basic math operators in Python:
\[+\ -\ /\ *\ \%\ \setminus \ <\ >\ <=\ >=\ //\ abs\ **\]

\ \\
Different types:
\begin{center}
\item Integers
\item Long
\item Float
\item Complex
\end{center}

\ \\
\centerline{INTERACTIVE DEMO}

\end{frame}




\begin{frame}{{\tiny \ \\\vspace{-13pt} \ \ \ \ \ \ \ \ \ \ \ Python Workshop}\\ \centerline{Truth}}
\vspace*{-50pt}

Special keywords: True, False

\ \\
Most types can be evaluated to a truth value...\\
True: any non-zero integer, characters, non-null strings, non-empty lists, non-empty dictionaries, ...\\
False: 0, [], ``", {},...

\ \\
Logic Operators:
\[ or\ and\ not\ !=\ ==\ <=\ >=\]


\end{frame}





\begin{frame}{{\tiny \ \\\vspace{-13pt} \ \ \ \ \ \ \ \ \ \ \ Python Workshop}\\ \centerline{Truth}}
\vspace*{-50pt}


BE CAREFUL:\\
\begin{center}
\texttt{True and 1\\
True == 'a'\\
True = 43\\
True == 1\\
True and "elephant"}\end{center}

\ \\
\centerline{DEMO}

\end{frame}






\begin{frame}{{\tiny \ \\\vspace{-13pt} \ \ \ \ \ \ \ \ \ \ \ Python Workshop}\\ \centerline{Characters and Strings}}
\vspace*{-50pt}


\begin{itemize}
\item Characters are enclosed within single quotes, e.g. 'a'
\item Strings use double quotes, e.g. "hola"
\item Combine strings with "+", e.g. "Super" + "man" = "Superman"
\item Access single characters using index, e.g. cat[2] = "t"
\end{itemize}



\ \\
\ \\
{\small Note: Just as in most things in computer science, indexing starts at 0 not 1!}

\end{frame}





\begin{frame}{{\tiny \ \\\vspace{-13pt} \ \ \ \ \ \ \ \ \ \ \ Python Workshop}\\ \centerline{Characters and Strings}}
\vspace*{-50pt}

Useful functions on strings:\\

\centerline{$append, +, *, [i], [i:j]$}

\ \\
\ \\
\centerline{INTERACTIVE DEMO}

\end{frame}





\begin{frame}{{\tiny \ \\\vspace{-13pt} \ \ \ \ \ \ \ \ \ \ \ Python Workshop}\\ \centerline{Back to Print}}
\vspace*{-50pt}


\begin{itemize}
\item Add ``," at end of print statement to keep cursor on same line
\item Can combine values using ``,", ``+", or inline format characters
\item Special characters: $\setminus n, \setminus\ ", \setminus\setminus, \setminus t, \setminus *$
\end{itemize}


\ \\
\ \\
\centerline{INTERACTIVE DEMO}




\end{frame}





\begin{frame}{{\tiny \ \\\vspace{-13pt} \ \ \ \ \ \ \ \ \ \ \ Python Workshop}\\ \centerline{Input}}
\vspace*{-50pt}

\begin{itemize}
\item Get input from console using \texttt{raw\_input("text")}
\item The argument for the function gets printed out and the output is whatever gets written by the user
\end{itemize}




\ \\
\ \\
\centerline{DEMO}


\end{frame}





\begin{frame}{{\tiny \ \\\vspace{-13pt} \ \ \ \ \ \ \ \ \ \ \ Python Workshop}\\ \centerline{Lists}}
\vspace*{-50pt}

\begin{itemize}
\item Python also supports Lists, e.g. $ ["a", "b"], [[[1],2],3], ["Edward", "Paul", "Suzie", "Not Nicole"]$
\item Lists are mutable
\item Unlike other languages, Python lists are ``untyped"
\item Useful operations on strings: pop,del, len, in, append, index, insert, remove(obj), reverse, sort
\end{itemize}




\ \\
\ \\
\centerline{INTERACTIVE DEMO}

\end{frame}








\begin{frame}{{\tiny \ \\\vspace{-13pt} \ \ \ \ \ \ \ \ \ \ \ Python Workshop}\\ \centerline{Dictionaries}}
\vspace*{-50pt}

\begin{itemize}
\item Python also supports Dictionaries by default, e.g. $me = {"name":"gabriel", "hands": "2", "t-shirt":"blue"} $
\item Dictionaries are also mutable
\item Access values through keys, e.g. $me["name"] = "gabriel"$
\item Useful operations: del, clear(), len(), has\_key(), items(), keys(), values(), update()
\end{itemize}


\ \\
\ \\
\centerline{INTERACTIVE DEMO}


\end{frame}






\begin{frame}{{\tiny \ \\\vspace{-13pt} \ \ \ \ \ \ \ \ \ \ \ Python Workshop}\\ \centerline{Tuples}}
\vspace*{-50pt}

\begin{itemize}
\item Tuples $(a1,...,an)$ are also supported in Python
\item Tuples are IMMUTABLE
\item You can still access the data, just not modify it
\end{itemize}


\ \\
\ \\
\centerline{INTERACTIVE DEMO}


\end{frame}





\begin{frame}{{\tiny \ \\\vspace{-13pt} \ \ \ \ \ \ \ \ \ \ \ Python Workshop}\\ \centerline{Functions}}
\vspace*{-50pt}

\begin{center}\texttt{def my\_function(inputs):\\
\ \ \ \ ....\\
\ \ \ \ $do\ stuff$\\
\ \ \ \ .... \\
\ \ \ \ return output
}\end{center}

\begin{itemize}
\item Functions are blocks of code with input and output
\item They are reusable structures
\item Functions are not run when encountered, have to be called
\item Functions have to already have been seen by the interpreter before being called
\end{itemize}


\ \\
\ \\
\centerline{DEMO}
%{\tiny first/second\_function.py}

\ \\
Note: In Python, tabs/spaces are super important to structure the code, as opposed to \{...\} or ``;" in other languages

\end{frame}




\begin{frame}{{\tiny \ \\\vspace{-13pt} \ \ \ \ \ \ \ \ \ \ \ Python Workshop}\\ \centerline{Back to Types}}
\vspace*{-50pt}

Not having explicit types can be fun and make code less heavy to read, but can be problematic as code \\



\ \\
\ \\
\centerline{INTERACTIVE DEMO}


\end{frame}








\begin{frame}{{\tiny \ \\\vspace{-13pt} \ \ \ \ \ \ \ \ \ \ \ Python Workshop}\\ \centerline{Memory}}
\vspace*{-50pt}

Some types in Python are more basic than others:\\
- Strings $\equiv$ Lists of characters\\
- Characters and Numbers passed by value instead of by Reference\\
- Dictionaries, Strings, Lists built recursively on other types\\



\ \\
\ \\
\centerline{INTERACTIVE DEMO}


\end{frame}





\begin{frame}{{\tiny \ \\\vspace{-13pt} \ \ \ \ \ \ \ \ \ \ \ Python Workshop}\\ \centerline{Files}}
\vspace*{-50pt}

\begin{itemize}
\item Open files with \texttt{variable = open(filename, flag)}
\item Read the files with: read(), readlines()
\item Write with \texttt{write()}
\item Close file with \texttt{close()} 
\end{itemize}


\ \\
\ \\
\centerline{INTERACTIVE DEMO}


\end{frame}








\begin{frame}{{\tiny \ \\\vspace{-13pt} \ \ \ \ \ \ \ \ \ \ \ Python Workshop}\\ \centerline{Conditionals}}
\vspace*{-50pt}
\begin{center}
\texttt{
if CONDITION:\\
\ \ \ \ $do\ stuff$\\
elif CONDITION:\\
\ \ \ \ $do\ stuff$\\
else:\\
\ \ \ \ $do\ stuff$\\}\end{center}

\begin{itemize}
\item Only \texttt{if} is obligatory
\item Interpreter tests a branch then either runs the inside code or skips to new branch
\end{itemize}


\ \\
\ \\
\centerline{INTERACTIVE DEMO}
WRITE IFF ESLSJFIE ELIF ILS CODE


\end{frame}





\begin{frame}{{\tiny \ \\\vspace{-13pt} \ \ \ \ \ \ \ \ \ \ \ Python Workshop}\\ \centerline{Loops}}
\vspace*{-50pt}

2 different kinds of loops in Python
\begin{itemize}
\item $For$ loops: Runs what is in the body once for each value in A\\
\texttt{for i in A:\\
\ \ \ \ $do\ stuff$\\}
\item $While$ loops: Runs what is in the body as long as the condition is true\\
\texttt{while CONDITION:\\
\ \ \ \ $do\ stuff$\\}
\end{itemize}

Use \texttt{break} to force exit from the loop


\ \\
\ \\
\centerline{INTERACTIVE DEMO}

\end{frame}



\begin{frame}{{\tiny \ \\\vspace{-13pt} \ \ \ \ \ \ \ \ \ \ \ Python Workshop}\\ \centerline{Regular Expressions}}
\vspace*{-50pt}

Super useful when dealing with text and words\\

Useful functions:
\begin{itemize}
\item \texttt{re.match(pattern, text, flags)}: searches the text for the pattern from the beginning of the text
\item \texttt{re.search(pattern, text, flags)}: searches the text for the pattern, anywhere in the text
\item \texttt{re.sub(pattern\_to\_match, pattern\_to\_sub, text)}: replace the first pattern by the second
\end{itemize}

The pattern in these cases has to be either of the form \texttt{r'pattern'} or you can use the \text{compile} function to turn a string into a pattern.


\ \\
\ \\
\centerline{TREE DEMO}


\end{frame}






\begin{frame}{{\tiny \ \\\vspace{-13pt} \ \ \ \ \ \ \ \ \ \ \ Python Workshop}\\ \centerline{Lambda}}
\vspace*{-50pt}

Python also supports anonymous functions of the style $\lambda x. f(x)$ written as \texttt{lambda x: f(x)}






\ \\
\ \\
\centerline{SEMANTICS DEMO}


\end{frame}





\begin{frame}{{\tiny \ \\\vspace{-13pt} \ \ \ \ \ \ \ \ \ \ \ Python Workshop}\\ \centerline{Resources}}
\vspace*{-50pt}

http://learnpythonthehardway.org/book/\\
https://web.stanford.edu/class/linguist278/\\
http://www.tutorialspoint.com/python/\\
https://docs.python.org/2/tutorial/



\end{frame}








\end{document}






\begin{frame}{{\tiny \ \\\vspace{-13pt} \ \ \ \ \ \ \ \ \ \ \ Python Workshop}\\\vspace{3pt} \centerline{Cat}}
\vspace*{-50pt}
%\topline
%\frametitle{\ \\ \[Hepa\]}
%\frametitle{\[Hepa\]}
%\frametitle{\begin{center}Hepa\end{center}}
%\begin{center}{\color{blue} \Large Hepa}\end{center}
%\begin{center}{\color{cyan} \Large Hepa}\end{center}
%\begin{textblock}{5}(7,1) Test\end{textblock}


non 

ne me

\textcolor{blue}{cat}

%{\color{RoyalBlue} cat}

dis

pas $what$ no non o no $kokokokokok$

what if

yeah 

why is the text so weird it's so far on the side

yeah 

she got you


\end{frame}














